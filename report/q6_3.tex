\section{PDE solver}

The graphs \ref{overlap} and \ref{synchro} show the speedup for both of the synchronization methods.

%\begin{figure}[!h]
 % \begin{center}
  %       \resizebox{160mm}{!}{\includegraphics{pic/graph_over_time.eps}}
  %\end{center}
  %\caption{Time needed using overlap}
%\end{figure}

\begin{figure}[H]
  \begin{center}
         \resizebox{160mm}{!}{\includegraphics{pic/graph_over.eps}}
  \end{center}
  \caption{Speedup using overlap}
  \label{overlap}
\end{figure}

%\begin{figure}[!h]
 % \begin{center}
  %       \resizebox{160mm}{!}{\includegraphics{pic/graph_synchro_time.eps}}
 % \end{center}
 % \caption{Time needed using synchronize}
%\end{figure}

\begin{figure}[!h]
  \begin{center}
         \resizebox{160mm}{!}{\includegraphics{pic/graph_synchro.eps}}
  \end{center}
  \caption{Speedup using synchronize}
  \label{synchro}
\end{figure}

As we can see, the speedup is a bit better using overlap instead of synchronize. Indeed, with this method, a thread blocks only if it is necessary where it waits for each iteation using the synchronize method. 

However, this is still a synchronization step, so the results look very similar. Indeed, the threads must be synchronized for each iteration in both methods but it takes less time in overlap than synchronize.

One can also observe that the overlap method gives better results with small problems size than the synchronize one. It makes sense because there is less synchronization overhead in the first case and such delays really matter when it comes to small problems. However, the overhead becomes quickly too important for such problems and the results drops, regardless of the synchronization method used. So, the problem size must be big enough to increase the number of threads without seeing the results decrease. 
